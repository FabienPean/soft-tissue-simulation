\chapter{Tutorial 5: Shape Matching}

In this tutorial we show how to match a volumetric liver liver shape matching example. The algorithm is a SOFA based implementation of the \emph{Physics based Shape Matching} approach by Suwelack et al. (\url{http://dx.doi.org/10.1118/1.4896021})

The python script example can be found in "/examples/PythonExamples/LiverShapeMatching/ShapeMatchingRunner.py". 

In order to run this example from the command line please first make sure that the MSML source is in your Python path:
\begin{lstlisting}[language=sh, breaklines=true]
$ export PYTHONPATH="$PYTHONPATH:/opt/msml/src"
\end{lstlisting}

Then change to the folder that contains the script

\begin{lstlisting}[language=sh, breaklines=true]
$ cd /opt/msml/examples/PythonExamples/LiverShapeMatching
\end{lstlisting}

and run it:

\begin{lstlisting}[language=sh, breaklines=true]
$ python ShapeMatchingRunner.py
\end{lstlisting}

ParaView can be used to view the result and inspect the meshes:
\begin{lstlisting}[language=sh, breaklines=true]
$ /opt/paraview/bin/paraview &
\end{lstlisting}

The method provides a stable registration process over a wide range of parameters. If you would like to play around with the most important parameters (i.e. charge density, stiffness and time step) you can change them in the LiverShapeMatching.msml.xml file.